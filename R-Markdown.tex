% Options for packages loaded elsewhere
\PassOptionsToPackage{unicode}{hyperref}
\PassOptionsToPackage{hyphens}{url}
%
\documentclass[
]{article}
\usepackage{amsmath,amssymb}
\usepackage{lmodern}
\usepackage{iftex}
\ifPDFTeX
  \usepackage[T1]{fontenc}
  \usepackage[utf8]{inputenc}
  \usepackage{textcomp} % provide euro and other symbols
\else % if luatex or xetex
  \usepackage{unicode-math}
  \defaultfontfeatures{Scale=MatchLowercase}
  \defaultfontfeatures[\rmfamily]{Ligatures=TeX,Scale=1}
\fi
% Use upquote if available, for straight quotes in verbatim environments
\IfFileExists{upquote.sty}{\usepackage{upquote}}{}
\IfFileExists{microtype.sty}{% use microtype if available
  \usepackage[]{microtype}
  \UseMicrotypeSet[protrusion]{basicmath} % disable protrusion for tt fonts
}{}
\makeatletter
\@ifundefined{KOMAClassName}{% if non-KOMA class
  \IfFileExists{parskip.sty}{%
    \usepackage{parskip}
  }{% else
    \setlength{\parindent}{0pt}
    \setlength{\parskip}{6pt plus 2pt minus 1pt}}
}{% if KOMA class
  \KOMAoptions{parskip=half}}
\makeatother
\usepackage{xcolor}
\usepackage[margin=1in]{geometry}
\usepackage{color}
\usepackage{fancyvrb}
\newcommand{\VerbBar}{|}
\newcommand{\VERB}{\Verb[commandchars=\\\{\}]}
\DefineVerbatimEnvironment{Highlighting}{Verbatim}{commandchars=\\\{\}}
% Add ',fontsize=\small' for more characters per line
\usepackage{framed}
\definecolor{shadecolor}{RGB}{248,248,248}
\newenvironment{Shaded}{\begin{snugshade}}{\end{snugshade}}
\newcommand{\AlertTok}[1]{\textcolor[rgb]{0.94,0.16,0.16}{#1}}
\newcommand{\AnnotationTok}[1]{\textcolor[rgb]{0.56,0.35,0.01}{\textbf{\textit{#1}}}}
\newcommand{\AttributeTok}[1]{\textcolor[rgb]{0.77,0.63,0.00}{#1}}
\newcommand{\BaseNTok}[1]{\textcolor[rgb]{0.00,0.00,0.81}{#1}}
\newcommand{\BuiltInTok}[1]{#1}
\newcommand{\CharTok}[1]{\textcolor[rgb]{0.31,0.60,0.02}{#1}}
\newcommand{\CommentTok}[1]{\textcolor[rgb]{0.56,0.35,0.01}{\textit{#1}}}
\newcommand{\CommentVarTok}[1]{\textcolor[rgb]{0.56,0.35,0.01}{\textbf{\textit{#1}}}}
\newcommand{\ConstantTok}[1]{\textcolor[rgb]{0.00,0.00,0.00}{#1}}
\newcommand{\ControlFlowTok}[1]{\textcolor[rgb]{0.13,0.29,0.53}{\textbf{#1}}}
\newcommand{\DataTypeTok}[1]{\textcolor[rgb]{0.13,0.29,0.53}{#1}}
\newcommand{\DecValTok}[1]{\textcolor[rgb]{0.00,0.00,0.81}{#1}}
\newcommand{\DocumentationTok}[1]{\textcolor[rgb]{0.56,0.35,0.01}{\textbf{\textit{#1}}}}
\newcommand{\ErrorTok}[1]{\textcolor[rgb]{0.64,0.00,0.00}{\textbf{#1}}}
\newcommand{\ExtensionTok}[1]{#1}
\newcommand{\FloatTok}[1]{\textcolor[rgb]{0.00,0.00,0.81}{#1}}
\newcommand{\FunctionTok}[1]{\textcolor[rgb]{0.00,0.00,0.00}{#1}}
\newcommand{\ImportTok}[1]{#1}
\newcommand{\InformationTok}[1]{\textcolor[rgb]{0.56,0.35,0.01}{\textbf{\textit{#1}}}}
\newcommand{\KeywordTok}[1]{\textcolor[rgb]{0.13,0.29,0.53}{\textbf{#1}}}
\newcommand{\NormalTok}[1]{#1}
\newcommand{\OperatorTok}[1]{\textcolor[rgb]{0.81,0.36,0.00}{\textbf{#1}}}
\newcommand{\OtherTok}[1]{\textcolor[rgb]{0.56,0.35,0.01}{#1}}
\newcommand{\PreprocessorTok}[1]{\textcolor[rgb]{0.56,0.35,0.01}{\textit{#1}}}
\newcommand{\RegionMarkerTok}[1]{#1}
\newcommand{\SpecialCharTok}[1]{\textcolor[rgb]{0.00,0.00,0.00}{#1}}
\newcommand{\SpecialStringTok}[1]{\textcolor[rgb]{0.31,0.60,0.02}{#1}}
\newcommand{\StringTok}[1]{\textcolor[rgb]{0.31,0.60,0.02}{#1}}
\newcommand{\VariableTok}[1]{\textcolor[rgb]{0.00,0.00,0.00}{#1}}
\newcommand{\VerbatimStringTok}[1]{\textcolor[rgb]{0.31,0.60,0.02}{#1}}
\newcommand{\WarningTok}[1]{\textcolor[rgb]{0.56,0.35,0.01}{\textbf{\textit{#1}}}}
\usepackage{graphicx}
\makeatletter
\def\maxwidth{\ifdim\Gin@nat@width>\linewidth\linewidth\else\Gin@nat@width\fi}
\def\maxheight{\ifdim\Gin@nat@height>\textheight\textheight\else\Gin@nat@height\fi}
\makeatother
% Scale images if necessary, so that they will not overflow the page
% margins by default, and it is still possible to overwrite the defaults
% using explicit options in \includegraphics[width, height, ...]{}
\setkeys{Gin}{width=\maxwidth,height=\maxheight,keepaspectratio}
% Set default figure placement to htbp
\makeatletter
\def\fps@figure{htbp}
\makeatother
\setlength{\emergencystretch}{3em} % prevent overfull lines
\providecommand{\tightlist}{%
  \setlength{\itemsep}{0pt}\setlength{\parskip}{0pt}}
\setcounter{secnumdepth}{-\maxdimen} % remove section numbering
\ifLuaTeX
  \usepackage{selnolig}  % disable illegal ligatures
\fi
\IfFileExists{bookmark.sty}{\usepackage{bookmark}}{\usepackage{hyperref}}
\IfFileExists{xurl.sty}{\usepackage{xurl}}{} % add URL line breaks if available
\urlstyle{same} % disable monospaced font for URLs
\hypersetup{
  pdftitle={Determination of Normal CI in 2nd and 3rd Trimester: Sample population},
  pdfauthor={Usman},
  hidelinks,
  pdfcreator={LaTeX via pandoc}}

\title{Determination of Normal CI in 2nd and 3rd Trimester: Sample
population}
\author{Usman}
\date{2023-10-22}

\begin{document}
\maketitle

{
\setcounter{tocdepth}{2}
\tableofcontents
}
\hypertarget{r-markdown}{%
\subsection{\texorpdfstring{\textbf{R
Markdown}}{R Markdown}}\label{r-markdown}}

\hypertarget{objective}{%
\section{\texorpdfstring{\textbf{Objective}}{Objective}}\label{objective}}

\begin{enumerate}
\def\labelenumi{\arabic{enumi}.}
\tightlist
\item
  Estimation of fetal cephalic index among Hausa/Fulani pregnant women
  in Rano LGA, Kano State.
\item
  Study the difference between 2nd and 3rd-trimester cephalic index
  values for both male and female
\item
  Assess possible association between sex and various obstetric
  parameters
\item
  Correlate the different fetal biometric parameters and gestational age
  to determine the best predictor
\end{enumerate}

\hypertarget{inclusion-criteria}{%
\subsection{\texorpdfstring{\textbf{Inclusion
criteria}:}{Inclusion criteria:}}\label{inclusion-criteria}}

\begin{itemize}
\tightlist
\item
  Singleton pregnancy. Dichorionic pregnancies with demise of 1 fetus in
  the first trimester were included
\item
  Formal pregnancy ultrasound performed at WCH at 28 to 33 weeks,
  including standard fetal biometry.
\item
  Morphology ultrasound performed at 17 to 22 weeks' gestation at WCH or
  another accredited images for review.
\end{itemize}

\begin{Shaded}
\begin{Highlighting}[]
\FunctionTok{library}\NormalTok{(readxl)}
\NormalTok{cephalic }\OtherTok{\textless{}{-}} \FunctionTok{read\_excel}\NormalTok{(}\StringTok{"\textasciitilde{}/Bluetooth Exchange Folder/sagagi2.xlsx"}\NormalTok{,}\AttributeTok{sheet =} \StringTok{"Sheet2"}\NormalTok{)}
\NormalTok{cephalic}
\end{Highlighting}
\end{Shaded}

\begin{verbatim}
## # A tibble: 631 x 9
##       ID SEX     BPD   OFD    FL    AC    CI EG.A  TRIMESTER
##    <dbl> <chr> <dbl> <dbl> <dbl> <dbl> <dbl> <chr> <chr>    
##  1     1 F        93   113    78   360    82 39+2  3RD      
##  2     2 M        91   108    72   338    84 37+0  3RD      
##  3     3 M        56    64    41   182    88 22+5  2ND      
##  4     4 F        52    62    38   183    84 22+0  2ND      
##  5     5 M        87   106    69   332    82 35+5  3RD      
##  6     6 F        83   101    64   296    82 33+1  3RD      
##  7     7 M        76    87    55   252    87 29+1  3RD      
##  8     8 F        69    83    53   242    83 27+4  3RD      
##  9     9 M        72    86    52   249    84 28+2  3RD      
## 10    10 M        71    83    49   236    86 27+2  3RD      
## # i 621 more rows
\end{verbatim}

\hypertarget{figures-code-chunks-and-logistic-regression-tables}{%
\subsection{Figures, Code chunks, and Logistic regression
tables}\label{figures-code-chunks-and-logistic-regression-tables}}

\hypertarget{figure-1}{%
\subsubsection{\texorpdfstring{\textbf{Figure
1}}{Figure 1}}\label{figure-1}}

\begin{Shaded}
\begin{Highlighting}[]
\FunctionTok{library}\NormalTok{(tidyverse)}
\end{Highlighting}
\end{Shaded}

\begin{verbatim}
## -- Attaching core tidyverse packages ------------------------ tidyverse 2.0.0 --
## v dplyr     1.1.1     v readr     2.1.4
## v forcats   1.0.0     v stringr   1.5.0
## v ggplot2   3.4.2     v tibble    3.2.1
## v lubridate 1.9.2     v tidyr     1.3.0
## v purrr     1.0.1     
## -- Conflicts ------------------------------------------ tidyverse_conflicts() --
## x dplyr::filter() masks stats::filter()
## x dplyr::lag()    masks stats::lag()
## i Use the ]8;;http://conflicted.r-lib.org/conflicted package]8;; to force all conflicts to become errors
\end{verbatim}

\begin{Shaded}
\begin{Highlighting}[]
\FunctionTok{library}\NormalTok{(ggpubr)}
\CommentTok{\#library(ggpubr) and library(tidyverse)}
\CommentTok{\# Vertical and Horizontal median representation of both variables}
\CommentTok{\#scatterplot showing correlation BPD and OFD at 2nd Trimester}
\NormalTok{meanbpd2}\OtherTok{\textless{}{-}}\NormalTok{cephalic}\SpecialCharTok{\%\textgreater{}\%}\FunctionTok{mutate}\NormalTok{(}\AttributeTok{EG.A=}\FunctionTok{str\_replace\_all}\NormalTok{(EG.A,}\StringTok{"[+]"}\NormalTok{,}\StringTok{"."}\NormalTok{),}\AttributeTok{TRIMESTER=}\FunctionTok{recode}\NormalTok{(TRIMESTER,}\StringTok{"2nd"}\OtherTok{=}\StringTok{"2ND"}\NormalTok{,}\StringTok{"3rd"}\OtherTok{=}\StringTok{"3RD"}\NormalTok{,}
  \StringTok{"2NDF"}\OtherTok{=}\StringTok{"2ND"}\NormalTok{))}\SpecialCharTok{\%\textgreater{}\%}\FunctionTok{filter}\NormalTok{(TRIMESTER}\SpecialCharTok{==}\StringTok{"2ND"}\NormalTok{)}\SpecialCharTok{\%\textgreater{}\%}\FunctionTok{summarise}\NormalTok{(}\AttributeTok{median\_val=}\FunctionTok{median}\NormalTok{(BPD))}
\NormalTok{meanbpd2}
\end{Highlighting}
\end{Shaded}

\begin{verbatim}
## # A tibble: 1 x 1
##   median_val
##        <dbl>
## 1         57
\end{verbatim}

\begin{Shaded}
\begin{Highlighting}[]
\NormalTok{meanofd2}\OtherTok{\textless{}{-}}\NormalTok{cephalic}\SpecialCharTok{\%\textgreater{}\%}\FunctionTok{mutate}\NormalTok{(}\AttributeTok{EG.A=}\FunctionTok{str\_replace\_all}\NormalTok{(EG.A,}\StringTok{"[+]"}\NormalTok{,}\StringTok{"."}\NormalTok{),}\AttributeTok{TRIMESTER=}\FunctionTok{recode}\NormalTok{(TRIMESTER,}\StringTok{"2nd"}\OtherTok{=}\StringTok{"2ND"}\NormalTok{,}\StringTok{"3rd"}\OtherTok{=}\StringTok{"3RD"}\NormalTok{,}
\StringTok{"2NDF"}\OtherTok{=}\StringTok{"2ND"}\NormalTok{))}\SpecialCharTok{\%\textgreater{}\%}\FunctionTok{filter}\NormalTok{(TRIMESTER}\SpecialCharTok{==}\StringTok{"2ND"}\NormalTok{)}\SpecialCharTok{\%\textgreater{}\%}\FunctionTok{summarise}\NormalTok{(}\AttributeTok{median\_val=}\FunctionTok{median}\NormalTok{(OFD))}
\NormalTok{meanofd2}
\end{Highlighting}
\end{Shaded}

\begin{verbatim}
## # A tibble: 1 x 1
##   median_val
##        <dbl>
## 1         68
\end{verbatim}

\begin{Shaded}
\begin{Highlighting}[]
\NormalTok{cephalic}\SpecialCharTok{\%\textgreater{}\%}\FunctionTok{mutate}\NormalTok{(}\AttributeTok{EG.A=}\FunctionTok{str\_replace\_all}\NormalTok{(EG.A,}\StringTok{"[+]"}\NormalTok{,}\StringTok{"."}\NormalTok{),}\AttributeTok{TRIMESTER=}\FunctionTok{recode}\NormalTok{(TRIMESTER,}\StringTok{"2nd"}\OtherTok{=}\StringTok{"2ND"}\NormalTok{,}\StringTok{"3rd"}\OtherTok{=}\StringTok{"3RD"}\NormalTok{,}\StringTok{"2NDF"}\OtherTok{=}\StringTok{"2ND"}\NormalTok{))}\SpecialCharTok{\%\textgreater{}\%}
  \FunctionTok{filter}\NormalTok{(TRIMESTER}\SpecialCharTok{==}\StringTok{"2ND"}\NormalTok{)}\SpecialCharTok{\%\textgreater{}\%}\FunctionTok{ggplot}\NormalTok{(}\FunctionTok{aes}\NormalTok{(BPD,OFD))}\SpecialCharTok{+}\FunctionTok{geom\_point}\NormalTok{()}\SpecialCharTok{+}\FunctionTok{theme\_minimal}\NormalTok{()}\SpecialCharTok{+}
  \FunctionTok{geom\_hline}\NormalTok{(}\AttributeTok{data=}\NormalTok{ meanofd2, }\FunctionTok{aes}\NormalTok{(}\AttributeTok{yintercept =}\NormalTok{ median\_val),}\AttributeTok{linetype=}\StringTok{"dashed"}\NormalTok{)}\SpecialCharTok{+}
  \FunctionTok{geom\_vline}\NormalTok{(}\AttributeTok{data=}\NormalTok{ meanbpd2, }\FunctionTok{aes}\NormalTok{(}\AttributeTok{xintercept =}\NormalTok{ median\_val),}\AttributeTok{linetype=}\StringTok{"longdash"}\NormalTok{)}\SpecialCharTok{+}\FunctionTok{geom\_smooth}\NormalTok{(}\AttributeTok{method =}\NormalTok{ lm,}\AttributeTok{se=}\ConstantTok{FALSE}\NormalTok{,}\AttributeTok{fullrange=}\ConstantTok{TRUE}\NormalTok{)}\SpecialCharTok{+}
  \FunctionTok{scale\_x\_log10}\NormalTok{()}\SpecialCharTok{+}\FunctionTok{stat\_cor}\NormalTok{(}\AttributeTok{method =} \StringTok{"spearman"}\NormalTok{)}
\end{Highlighting}
\end{Shaded}

\begin{verbatim}
## `geom_smooth()` using formula = 'y ~ x'
\end{verbatim}

\includegraphics{R-Markdown_files/figure-latex/pressure-1.pdf}

\textbf{Figure 1}: Scatter plot of OFD versus BPD at the morphology
scan; points have been `jittered' horizontally and vertically to
separate identical values; the dash lines are the median values for OFD
and BPD; the solid blue line represents the correlation line

\hypertarget{figure-2}{%
\subsubsection{\texorpdfstring{\textbf{Figure
2}}{Figure 2}}\label{figure-2}}

\begin{Shaded}
\begin{Highlighting}[]
\NormalTok{meanbpd3}\OtherTok{\textless{}{-}}\NormalTok{cephalic}\SpecialCharTok{\%\textgreater{}\%}\FunctionTok{mutate}\NormalTok{(}\AttributeTok{EG.A=}\FunctionTok{str\_replace\_all}\NormalTok{(EG.A,}\StringTok{"[+]"}\NormalTok{,}\StringTok{"."}\NormalTok{),}
  \AttributeTok{TRIMESTER=}\FunctionTok{recode}\NormalTok{(TRIMESTER,}\StringTok{"2nd"}\OtherTok{=}\StringTok{"2ND"}\NormalTok{,}\StringTok{"3rd"}\OtherTok{=}\StringTok{"3RD"}\NormalTok{,}\StringTok{"2NDF"}\OtherTok{=}\StringTok{"2ND"}\NormalTok{))}\SpecialCharTok{\%\textgreater{}\%}
  \FunctionTok{filter}\NormalTok{(TRIMESTER}\SpecialCharTok{==}\StringTok{"3RD"}\NormalTok{)}\SpecialCharTok{\%\textgreater{}\%}
  \FunctionTok{summarise}\NormalTok{(}\AttributeTok{median\_val=}\FunctionTok{median}\NormalTok{(BPD))}

\NormalTok{meanofd3}\OtherTok{\textless{}{-}}\NormalTok{cephalic}\SpecialCharTok{\%\textgreater{}\%}\FunctionTok{mutate}\NormalTok{(}\AttributeTok{EG.A=}\FunctionTok{str\_replace\_all}\NormalTok{(EG.A,}\StringTok{"[+]"}\NormalTok{,}\StringTok{"."}\NormalTok{),}
                        \AttributeTok{TRIMESTER=}\FunctionTok{recode}\NormalTok{(TRIMESTER,}\StringTok{"2nd"}\OtherTok{=}\StringTok{"2ND"}\NormalTok{,}\StringTok{"3rd"}\OtherTok{=}\StringTok{"3RD"}\NormalTok{,}
                        \StringTok{"2NDF"}\OtherTok{=}\StringTok{"2ND"}\NormalTok{))}\SpecialCharTok{\%\textgreater{}\%}\FunctionTok{filter}\NormalTok{(TRIMESTER}\SpecialCharTok{==}\StringTok{"3RD"}\NormalTok{)}\SpecialCharTok{\%\textgreater{}\%}
                         \FunctionTok{summarise}\NormalTok{(}\AttributeTok{median\_val=}\FunctionTok{mean}\NormalTok{(OFD))}


\NormalTok{cephalic}\SpecialCharTok{\%\textgreater{}\%}\FunctionTok{mutate}\NormalTok{(}\AttributeTok{EG.A=}\FunctionTok{str\_replace\_all}\NormalTok{(EG.A,}\StringTok{"[+]"}\NormalTok{,}\StringTok{"."}\NormalTok{),}\AttributeTok{TRIMESTER=}\FunctionTok{recode}\NormalTok{(TRIMESTER,}\StringTok{"2nd"}\OtherTok{=}\StringTok{"2ND"}\NormalTok{,}\StringTok{"3rd"}\OtherTok{=}\StringTok{"3RD"}\NormalTok{,}\StringTok{"2NDF"}\OtherTok{=}\StringTok{"2ND"}\NormalTok{))}\SpecialCharTok{\%\textgreater{}\%}
  \FunctionTok{filter}\NormalTok{(TRIMESTER}\SpecialCharTok{==}\StringTok{"3RD"}\NormalTok{)}\SpecialCharTok{\%\textgreater{}\%}\FunctionTok{ggplot}\NormalTok{(}\FunctionTok{aes}\NormalTok{(BPD,OFD))}\SpecialCharTok{+}
  \FunctionTok{geom\_point}\NormalTok{()}\SpecialCharTok{+}\FunctionTok{theme\_minimal}\NormalTok{()}\SpecialCharTok{+}
  \FunctionTok{geom\_hline}\NormalTok{(}\AttributeTok{data=}\NormalTok{ meanofd3, }\FunctionTok{aes}\NormalTok{(}\AttributeTok{yintercept =}\NormalTok{ median\_val),}\AttributeTok{linetype=}\StringTok{"dashed"}\NormalTok{)}\SpecialCharTok{+}
  \FunctionTok{geom\_vline}\NormalTok{(}\AttributeTok{data=}\NormalTok{ meanbpd3, }\FunctionTok{aes}\NormalTok{(}\AttributeTok{xintercept =}\NormalTok{ median\_val),}\AttributeTok{linetype=}\StringTok{"longdash"}\NormalTok{)}\SpecialCharTok{+}
  \FunctionTok{geom\_smooth}\NormalTok{(}\AttributeTok{method =}\NormalTok{ lm,}\AttributeTok{se=}\ConstantTok{FALSE}\NormalTok{,}\AttributeTok{fullrange=}\ConstantTok{TRUE}\NormalTok{)}\SpecialCharTok{+}\FunctionTok{scale\_x\_log10}\NormalTok{()}\SpecialCharTok{+}\FunctionTok{scale\_y\_log10}\NormalTok{()}\SpecialCharTok{+}
  \FunctionTok{stat\_cor}\NormalTok{(}\AttributeTok{method =} \StringTok{"spearman"}\NormalTok{)}
\end{Highlighting}
\end{Shaded}

\begin{verbatim}
## `geom_smooth()` using formula = 'y ~ x'
\end{verbatim}

\includegraphics{R-Markdown_files/figure-latex/unnamed-chunk-2-1.pdf}

\textbf{Figure 2}: Scatter plot of OFD versus BPD at the growth scan;
points have been `jittered' horizontally and vertically to separate
identical values; the dash lines are the median values for OFD and BPD;
the solid blue line represents the correlation line

\hypertarget{figure-3}{%
\subsubsection{\texorpdfstring{\textbf{Figure
3}}{Figure 3}}\label{figure-3}}

\begin{Shaded}
\begin{Highlighting}[]
\CommentTok{\#Scatter plot of the cephalic index at the growth scan versus the cephalic index at the morphology scan}
\NormalTok{ci\_morho}\OtherTok{\textless{}{-}}\NormalTok{cephalic}\SpecialCharTok{\%\textgreater{}\%}\FunctionTok{mutate}\NormalTok{(}\AttributeTok{EG.A1=}\FunctionTok{str\_replace\_all}\NormalTok{(EG.A,}\StringTok{"[+]"}\NormalTok{,}\StringTok{"."}\NormalTok{),}
  \AttributeTok{TRIMESTER=}\FunctionTok{recode}\NormalTok{(TRIMESTER,}\StringTok{"2nd"}\OtherTok{=}\StringTok{"2ND"}\NormalTok{,}\StringTok{"3rd"}\OtherTok{=}\StringTok{"3RD"}\NormalTok{,}\StringTok{"2NDF"}\OtherTok{=}\StringTok{"2ND"}\NormalTok{))}\SpecialCharTok{\%\textgreater{}\%}
  \FunctionTok{mutate}\NormalTok{(}\AttributeTok{CI3=}\NormalTok{(BPD}\SpecialCharTok{/}\NormalTok{OFD)}\SpecialCharTok{*}\DecValTok{100}\NormalTok{)}\SpecialCharTok{\%\textgreater{}\%}\FunctionTok{filter}\NormalTok{(TRIMESTER}\SpecialCharTok{==}\StringTok{"3RD"}\NormalTok{)}\SpecialCharTok{\%\textgreater{}\%}
  \FunctionTok{select}\NormalTok{(TRIMESTER,CI3,EG.A1)}
\NormalTok{ci\_growth}\OtherTok{\textless{}{-}}\NormalTok{cephalic}\SpecialCharTok{\%\textgreater{}\%}\FunctionTok{mutate}\NormalTok{(}\AttributeTok{EG.A=}\FunctionTok{str\_replace\_all}\NormalTok{(EG.A,}\StringTok{"[+]"}\NormalTok{,}\StringTok{"."}\NormalTok{),}
  \AttributeTok{TRIMESTER=}\FunctionTok{recode}\NormalTok{(TRIMESTER,}\StringTok{"2nd"}\OtherTok{=}\StringTok{"2ND"}\NormalTok{,}\StringTok{"3rd"}\OtherTok{=}\StringTok{"3RD"}\NormalTok{,}\StringTok{"2NDF"}\OtherTok{=}\StringTok{"2ND"}\NormalTok{))}\SpecialCharTok{\%\textgreater{}\%}
  \FunctionTok{mutate}\NormalTok{(}\AttributeTok{CI2=}\NormalTok{(BPD}\SpecialCharTok{/}\NormalTok{OFD)}\SpecialCharTok{*}\DecValTok{100}\NormalTok{)}\SpecialCharTok{\%\textgreater{}\%}\FunctionTok{filter}\NormalTok{(TRIMESTER}\SpecialCharTok{==}\StringTok{"2ND"}\NormalTok{)}

\NormalTok{ci\_medianM}\OtherTok{\textless{}{-}}\FunctionTok{tibble}\NormalTok{(ci\_growth,}\AttributeTok{CI3=}\NormalTok{ci\_morho[}\SpecialCharTok{{-}}\FunctionTok{c}\NormalTok{(}\DecValTok{199}\SpecialCharTok{:}\DecValTok{433}\NormalTok{),]}\SpecialCharTok{$}\NormalTok{CI3)}\SpecialCharTok{\%\textgreater{}\%}
  \FunctionTok{summarise}\NormalTok{(}\AttributeTok{median\_val=}\FunctionTok{median}\NormalTok{(CI2))}

\NormalTok{ci\_medianG}\OtherTok{\textless{}{-}}\FunctionTok{tibble}\NormalTok{(ci\_growth,}\AttributeTok{CI3=}\NormalTok{ci\_morho[}\SpecialCharTok{{-}}\FunctionTok{c}\NormalTok{(}\DecValTok{199}\SpecialCharTok{:}\DecValTok{433}\NormalTok{),]}\SpecialCharTok{$}\NormalTok{CI3)}\SpecialCharTok{\%\textgreater{}\%}
  \FunctionTok{summarise}\NormalTok{(}\AttributeTok{median\_val=}\FunctionTok{median}\NormalTok{(CI3))}

\FunctionTok{tibble}\NormalTok{(ci\_growth,}\AttributeTok{CI3=}\NormalTok{ci\_morho[}\SpecialCharTok{{-}}\FunctionTok{c}\NormalTok{(}\DecValTok{199}\SpecialCharTok{:}\DecValTok{433}\NormalTok{),]}\SpecialCharTok{$}\NormalTok{CI3)}\SpecialCharTok{\%\textgreater{}\%}\FunctionTok{ggplot}\NormalTok{(}\FunctionTok{aes}\NormalTok{(CI2,CI3))}\SpecialCharTok{+}
  \FunctionTok{geom\_point}\NormalTok{()}\SpecialCharTok{+}\FunctionTok{theme\_minimal}\NormalTok{()}\SpecialCharTok{+}\FunctionTok{scale\_x\_log10}\NormalTok{()}\SpecialCharTok{+}\FunctionTok{scale\_y\_log10}\NormalTok{()}\SpecialCharTok{+}
  \FunctionTok{labs}\NormalTok{(}\AttributeTok{x=}\StringTok{"CI at morphology scan"}\NormalTok{,}\AttributeTok{y=}\StringTok{"CI at growth scan"}\NormalTok{)}\SpecialCharTok{+}
  \FunctionTok{geom\_hline}\NormalTok{(}\AttributeTok{data=}\NormalTok{ ci\_medianG, }\FunctionTok{aes}\NormalTok{(}\AttributeTok{yintercept =}\NormalTok{ median\_val),}\AttributeTok{linetype=}\StringTok{"dashed"}\NormalTok{)}\SpecialCharTok{+}
  \FunctionTok{geom\_vline}\NormalTok{(}\AttributeTok{data=}\NormalTok{ ci\_medianM, }\FunctionTok{aes}\NormalTok{(}\AttributeTok{xintercept =}\NormalTok{ median\_val),}\AttributeTok{linetype=}\StringTok{"longdash"}\NormalTok{)}\SpecialCharTok{+}
  \FunctionTok{geom\_smooth}\NormalTok{(}\AttributeTok{method =}\NormalTok{ lm,}\AttributeTok{se=}\ConstantTok{FALSE}\NormalTok{,}\AttributeTok{fullrange=}\ConstantTok{TRUE}\NormalTok{)}\SpecialCharTok{+}\FunctionTok{stat\_cor}\NormalTok{(}\AttributeTok{method =} \StringTok{"spearman"}\NormalTok{)}
\end{Highlighting}
\end{Shaded}

\begin{verbatim}
## `geom_smooth()` using formula = 'y ~ x'
\end{verbatim}

\includegraphics{R-Markdown_files/figure-latex/unnamed-chunk-3-1.pdf}

\textbf{Figure 3}: Scatter plot of the cephalic index at the growth scan
versus the cephalic index at the morphology scan, showing little
difference between the 2 groups.

\hypertarget{figure-4}{%
\subsubsection{\texorpdfstring{\textbf{Figure
4}}{Figure 4}}\label{figure-4}}

\begin{Shaded}
\begin{Highlighting}[]
\NormalTok{ci\_growth1}\OtherTok{\textless{}{-}}\NormalTok{cephalic}\SpecialCharTok{\%\textgreater{}\%}\FunctionTok{mutate}\NormalTok{(}\AttributeTok{EG.A=}\FunctionTok{str\_replace\_all}\NormalTok{(EG.A,}\StringTok{"[+]"}\NormalTok{,}\StringTok{"."}\NormalTok{),}
  \AttributeTok{TRIMESTER=}\FunctionTok{recode}\NormalTok{(TRIMESTER,}\StringTok{"2nd"}\OtherTok{=}\StringTok{"2ND"}\NormalTok{,}\StringTok{"3rd"}\OtherTok{=}\StringTok{"3RD"}\NormalTok{,}\StringTok{"2NDF"}\OtherTok{=}\StringTok{"2ND"}\NormalTok{))}\SpecialCharTok{\%\textgreater{}\%}
  \FunctionTok{mutate}\NormalTok{(}\AttributeTok{CI2=}\NormalTok{(BPD}\SpecialCharTok{/}\NormalTok{OFD)}\SpecialCharTok{*}\DecValTok{100}\NormalTok{)}\SpecialCharTok{\%\textgreater{}\%}\FunctionTok{filter}\NormalTok{(TRIMESTER}\SpecialCharTok{==}\StringTok{"2ND"}\NormalTok{)}\SpecialCharTok{\%\textgreater{}\%}
  \FunctionTok{mutate}\NormalTok{(}\AttributeTok{CI3=}\NormalTok{CI2)}\SpecialCharTok{\%\textgreater{}\%}\FunctionTok{select}\NormalTok{(TRIMESTER,CI3,EG.A)}

\NormalTok{ci\_morpho}\OtherTok{\textless{}{-}}\NormalTok{cephalic}\SpecialCharTok{\%\textgreater{}\%}\FunctionTok{mutate}\NormalTok{(}\AttributeTok{EG.A1=}\FunctionTok{str\_replace\_all}\NormalTok{(EG.A,}\StringTok{"[+]"}\NormalTok{,}\StringTok{"."}\NormalTok{),}
  \AttributeTok{TRIMESTER=}\FunctionTok{recode}\NormalTok{(TRIMESTER,}\StringTok{"2nd"}\OtherTok{=}\StringTok{"2ND"}\NormalTok{,}\StringTok{"3rd"}\OtherTok{=}\StringTok{"3RD"}\NormalTok{,}\StringTok{"2NDF"}\OtherTok{=}\StringTok{"2ND"}\NormalTok{))}\SpecialCharTok{\%\textgreater{}\%}
  \FunctionTok{mutate}\NormalTok{(}\AttributeTok{CI2=}\NormalTok{(BPD}\SpecialCharTok{/}\NormalTok{OFD)}\SpecialCharTok{*}\DecValTok{100}\NormalTok{)}\SpecialCharTok{\%\textgreater{}\%}\FunctionTok{filter}\NormalTok{(TRIMESTER}\SpecialCharTok{==}\StringTok{"3RD"}\NormalTok{)}\SpecialCharTok{\%\textgreater{}\%}
  \FunctionTok{select}\NormalTok{(TRIMESTER,CI2,EG.A1)}

\NormalTok{ci\_morpho}\SpecialCharTok{$}\NormalTok{EG.A1}\OtherTok{\textless{}{-}}\FunctionTok{as.numeric}\NormalTok{(ci\_morpho}\SpecialCharTok{$}\NormalTok{EG.A1)}
\NormalTok{ci\_growth1}\SpecialCharTok{$}\NormalTok{EG.A}\OtherTok{\textless{}{-}}\FunctionTok{as.numeric}\NormalTok{(ci\_growth1}\SpecialCharTok{$}\NormalTok{EG.A)}

\NormalTok{new\_data}\OtherTok{\textless{}{-}}\FunctionTok{tibble}\NormalTok{(ci\_growth1[,}\SpecialCharTok{{-}}\DecValTok{1}\NormalTok{],ci\_morpho[}\SpecialCharTok{{-}}\FunctionTok{c}\NormalTok{(}\DecValTok{199}\SpecialCharTok{:}\DecValTok{433}\NormalTok{),])}\SpecialCharTok{\%\textgreater{}\%}
  \FunctionTok{mutate}\NormalTok{(}\AttributeTok{gest\_dif\_wk=}\NormalTok{ ci\_morpho[}\SpecialCharTok{{-}}\FunctionTok{c}\NormalTok{(}\DecValTok{199}\SpecialCharTok{:}\DecValTok{433}\NormalTok{),]}\SpecialCharTok{$}\NormalTok{EG.A1}\SpecialCharTok{{-}}\NormalTok{ci\_growth1[,}\SpecialCharTok{{-}}\DecValTok{1}\NormalTok{]}\SpecialCharTok{$}\NormalTok{EG.A,}\AttributeTok{ci\_change=}\NormalTok{CI3}\SpecialCharTok{{-}}\NormalTok{CI2)}

\NormalTok{new\_data}\SpecialCharTok{\%\textgreater{}\%}\FunctionTok{ggplot}\NormalTok{(}\FunctionTok{aes}\NormalTok{(gest\_dif\_wk,ci\_change))}\SpecialCharTok{+}
  \FunctionTok{geom\_point}\NormalTok{()}\SpecialCharTok{+}\FunctionTok{geom\_smooth}\NormalTok{(}\AttributeTok{method =}\NormalTok{ lm,}\AttributeTok{se=}\ConstantTok{FALSE}\NormalTok{,}\AttributeTok{fullrange=}\ConstantTok{TRUE}\NormalTok{)}\SpecialCharTok{+}
  \FunctionTok{stat\_cor}\NormalTok{(}\AttributeTok{method =} \StringTok{"spearman"}\NormalTok{)}\SpecialCharTok{+}\FunctionTok{scale\_y\_continuous}\NormalTok{(}\AttributeTok{limits =} \FunctionTok{c}\NormalTok{(}\SpecialCharTok{{-}}\DecValTok{10}\NormalTok{,}\FloatTok{10.3}\NormalTok{))}\SpecialCharTok{+}
  \FunctionTok{scale\_x\_continuous}\NormalTok{(}\AttributeTok{limits =} \FunctionTok{c}\NormalTok{(}\DecValTok{0}\NormalTok{,}\DecValTok{20}\NormalTok{))}\SpecialCharTok{+}\FunctionTok{geom\_hline}\NormalTok{(}\AttributeTok{yintercept =} \DecValTok{0}\NormalTok{, }\AttributeTok{linetype=}\StringTok{"dashed"}\NormalTok{)}\SpecialCharTok{+}
  \FunctionTok{theme\_minimal}\NormalTok{()}\SpecialCharTok{+}\FunctionTok{labs}\NormalTok{(}\AttributeTok{x=}\StringTok{"difference in Gestation (week)"}\NormalTok{, }\AttributeTok{y=} \StringTok{"Change in CI"}\NormalTok{)}
\end{Highlighting}
\end{Shaded}

\begin{verbatim}
## `geom_smooth()` using formula = 'y ~ x'
\end{verbatim}

\includegraphics{R-Markdown_files/figure-latex/unnamed-chunk-4-1.pdf}

\textbf{Figure 4}: Scatter plot of the change in cephalic index between
growth and morphology scans versus the corresponding change in gestation
age; the blue solid line shows the linear regression line; the dash tick
line is a reference line for zero change in CI.

\hypertarget{figure-5}{%
\subsubsection{\texorpdfstring{\textbf{Figure
5}}{Figure 5}}\label{figure-5}}

\begin{Shaded}
\begin{Highlighting}[]
\CommentTok{\#comparing means and adding to our figure}
\NormalTok{my\_comparisons}\OtherTok{\textless{}{-}}\FunctionTok{list}\NormalTok{(}\FunctionTok{c}\NormalTok{(}\StringTok{"2ND"}\NormalTok{,}\StringTok{"3RD"}\NormalTok{))}

\NormalTok{cephalic}\SpecialCharTok{\%\textgreater{}\%}\FunctionTok{mutate}\NormalTok{(}\AttributeTok{TRIMESTER=}\FunctionTok{recode}\NormalTok{(TRIMESTER,}\StringTok{"2nd"}\OtherTok{=}\StringTok{"2ND"}\NormalTok{,}\StringTok{"3rd"}\OtherTok{=}\StringTok{"3RD"}\NormalTok{,}\StringTok{"2NDF"}\OtherTok{=}\StringTok{"2ND"}\NormalTok{))}\SpecialCharTok{\%\textgreater{}\%}
  \FunctionTok{ggplot}\NormalTok{(}\FunctionTok{aes}\NormalTok{(TRIMESTER,CI))}\SpecialCharTok{+}\FunctionTok{geom\_boxplot}\NormalTok{()}\SpecialCharTok{+}\FunctionTok{geom\_point}\NormalTok{(}\AttributeTok{position=}\FunctionTok{position\_jitter}\NormalTok{())}\SpecialCharTok{+}
  \FunctionTok{theme\_minimal}\NormalTok{()}\SpecialCharTok{+}\FunctionTok{stat\_compare\_means}\NormalTok{(}\AttributeTok{comparisons =}\NormalTok{ my\_comparisons)}\SpecialCharTok{+}\FunctionTok{facet\_wrap}\NormalTok{(}\SpecialCharTok{\textasciitilde{}}\NormalTok{SEX)}
\end{Highlighting}
\end{Shaded}

\includegraphics{R-Markdown_files/figure-latex/unnamed-chunk-5-1.pdf}

\textbf{Figure 5}: Comparison between group relative to gender

\hypertarget{figure-6-using-ggpairs}{%
\subsubsection{\texorpdfstring{\textbf{Figure 6}: using
`ggpairs()'}{Figure 6: using `ggpairs()'}}\label{figure-6-using-ggpairs}}

\begin{Shaded}
\begin{Highlighting}[]
\FunctionTok{library}\NormalTok{(GGally)}
\end{Highlighting}
\end{Shaded}

\begin{verbatim}
## Registered S3 method overwritten by 'GGally':
##   method from   
##   +.gg   ggplot2
\end{verbatim}

\begin{Shaded}
\begin{Highlighting}[]
\CommentTok{\#pairwise comparisons between morphology and growth variables}
\NormalTok{cephalic}\SpecialCharTok{\%\textgreater{}\%}\FunctionTok{mutate}\NormalTok{(}\AttributeTok{TRIMESTER=}\FunctionTok{recode}\NormalTok{(TRIMESTER,}\StringTok{"2nd"}\OtherTok{=}\StringTok{"2ND"}\NormalTok{,}\StringTok{"3rd"}\OtherTok{=}\StringTok{"3RD"}\NormalTok{,}\StringTok{"2NDF"}\OtherTok{=}\StringTok{"2ND"}\NormalTok{))}\SpecialCharTok{\%\textgreater{}\%}
  \FunctionTok{ggpairs}\NormalTok{(}\AttributeTok{columns=}\DecValTok{3}\SpecialCharTok{:}\DecValTok{7}\NormalTok{,}\FunctionTok{aes}\NormalTok{(}\AttributeTok{col=}\NormalTok{TRIMESTER))}\SpecialCharTok{+}\FunctionTok{theme}\NormalTok{(}\AttributeTok{panel.grid.major =} \FunctionTok{element\_blank}\NormalTok{(),}
  \AttributeTok{axis.line =} \FunctionTok{element\_line}\NormalTok{(}\AttributeTok{colour =} \StringTok{"black"}\NormalTok{))}\SpecialCharTok{+}
  \FunctionTok{theme}\NormalTok{(}\AttributeTok{strip.background =} \FunctionTok{element\_rect}\NormalTok{(}\AttributeTok{fill =} \StringTok{"white"}\NormalTok{),}\AttributeTok{strip.placement =} \StringTok{"outside"}\NormalTok{)}\SpecialCharTok{+}
  \FunctionTok{theme}\NormalTok{(}\AttributeTok{axis.text.x =} \FunctionTok{element\_blank}\NormalTok{())}\SpecialCharTok{+}\FunctionTok{scale\_color\_manual}\NormalTok{(}\AttributeTok{values=}\FunctionTok{c}\NormalTok{(}\StringTok{"blue"}\NormalTok{,}\StringTok{"black"}\NormalTok{))}\SpecialCharTok{+}
  \FunctionTok{scale\_fill\_manual}\NormalTok{(}\AttributeTok{values=}\FunctionTok{c}\NormalTok{(}\StringTok{"blue"}\NormalTok{,}\StringTok{"black"}\NormalTok{))}
\end{Highlighting}
\end{Shaded}

\includegraphics{R-Markdown_files/figure-latex/unnamed-chunk-6-1.pdf}

\textbf{Figure 6} Correlation matrix

\hypertarget{logistic-regression-analysis}{%
\subsubsection{\texorpdfstring{\textbf{Logistic regression
analysis}}{Logistic regression analysis}}\label{logistic-regression-analysis}}

\begin{Shaded}
\begin{Highlighting}[]
\CommentTok{\#logistic regression analysis}
\NormalTok{cephalic}\SpecialCharTok{$}\NormalTok{sex}\OtherTok{\textless{}{-}}\NormalTok{cephalic}\SpecialCharTok{$}\NormalTok{SEX}
\NormalTok{cephalic}\SpecialCharTok{$}\NormalTok{logit}\OtherTok{\textless{}{-}}\NormalTok{cephalic}\SpecialCharTok{$}\NormalTok{TRIMESTER}
\NormalTok{cephalic[cephalic}\SpecialCharTok{$}\NormalTok{logit}\SpecialCharTok{==}\StringTok{"2ND"}\SpecialCharTok{|}\NormalTok{cephalic}\SpecialCharTok{$}\NormalTok{logit}\SpecialCharTok{==}\StringTok{"2nd"}\SpecialCharTok{|}\NormalTok{cephalic}\SpecialCharTok{$}\NormalTok{logit}\SpecialCharTok{==}\StringTok{"2NDF"}\NormalTok{,]}\SpecialCharTok{$}\NormalTok{logit}\OtherTok{\textless{}{-}}\StringTok{"0"}
\NormalTok{cephalic[cephalic}\SpecialCharTok{$}\NormalTok{logit}\SpecialCharTok{==}\StringTok{"3RD"}\SpecialCharTok{|}\NormalTok{cephalic}\SpecialCharTok{$}\NormalTok{logit}\SpecialCharTok{==}\StringTok{"3rd"}\NormalTok{,]}\SpecialCharTok{$}\NormalTok{logit}\OtherTok{\textless{}{-}}\StringTok{"1"}
\NormalTok{cephalic[cephalic}\SpecialCharTok{$}\NormalTok{sex}\SpecialCharTok{==}\StringTok{"F"}\NormalTok{,]}\SpecialCharTok{$}\NormalTok{sex}\OtherTok{\textless{}{-}}\StringTok{"0"}
\NormalTok{cephalic[cephalic}\SpecialCharTok{$}\NormalTok{sex}\SpecialCharTok{==}\StringTok{"M"}\NormalTok{,]}\SpecialCharTok{$}\NormalTok{sex}\OtherTok{\textless{}{-}}\StringTok{"1"}
\NormalTok{cephalic}\SpecialCharTok{$}\NormalTok{logit}\OtherTok{\textless{}{-}}\FunctionTok{as.factor}\NormalTok{(cephalic}\SpecialCharTok{$}\NormalTok{logit)}
\FunctionTok{glm}\NormalTok{(logit}\SpecialCharTok{\textasciitilde{}}\NormalTok{sex, }\AttributeTok{data=}\NormalTok{cephalic,}\AttributeTok{family =} \StringTok{"binomial"}\NormalTok{)}
\end{Highlighting}
\end{Shaded}

\begin{verbatim}
## 
## Call:  glm(formula = logit ~ sex, family = "binomial", data = cephalic)
## 
## Coefficients:
## (Intercept)         sex1  
##      0.6983       0.1589  
## 
## Degrees of Freedom: 630 Total (i.e. Null);  629 Residual
## Null Deviance:       785.1 
## Residual Deviance: 784.2     AIC: 788.2
\end{verbatim}

\begin{Shaded}
\begin{Highlighting}[]
\FunctionTok{summary}\NormalTok{(}\FunctionTok{glm}\NormalTok{(logit}\SpecialCharTok{\textasciitilde{}}\NormalTok{sex, }\AttributeTok{data=}\NormalTok{cephalic,}\AttributeTok{family =} \StringTok{"binomial"}\NormalTok{))}
\end{Highlighting}
\end{Shaded}

\begin{verbatim}
## 
## Call:
## glm(formula = logit ~ sex, family = "binomial", data = cephalic)
## 
## Deviance Residuals: 
##     Min       1Q   Median       3Q      Max  
## -1.5562  -1.4846   0.8411   0.8986   0.8986  
## 
## Coefficients:
##             Estimate Std. Error z value Pr(>|z|)    
## (Intercept)   0.6983     0.1242   5.620 1.91e-08 ***
## sex1          0.1589     0.1719   0.924    0.355    
## ---
## Signif. codes:  0 '***' 0.001 '**' 0.01 '*' 0.05 '.' 0.1 ' ' 1
## 
## (Dispersion parameter for binomial family taken to be 1)
## 
##     Null deviance: 785.09  on 630  degrees of freedom
## Residual deviance: 784.23  on 629  degrees of freedom
## AIC: 788.23
## 
## Number of Fisher Scoring iterations: 4
\end{verbatim}

\begin{Shaded}
\begin{Highlighting}[]
\FunctionTok{summary}\NormalTok{(}\FunctionTok{glm}\NormalTok{(logit}\SpecialCharTok{\textasciitilde{}}\NormalTok{SEX}\SpecialCharTok{+}\NormalTok{BPD}\SpecialCharTok{+}\NormalTok{OFD}\SpecialCharTok{+}\NormalTok{FL}\SpecialCharTok{+}\NormalTok{AC, }\AttributeTok{data=}\NormalTok{cephalic,}\AttributeTok{family =} \StringTok{"binomial"}\NormalTok{))}
\end{Highlighting}
\end{Shaded}

\begin{verbatim}
## 
## Call:
## glm(formula = logit ~ SEX + BPD + OFD + FL + AC, family = "binomial", 
##     data = cephalic)
## 
## Deviance Residuals: 
##     Min       1Q   Median       3Q      Max  
## -4.4158  -0.0002   0.0001   0.0081   1.2359  
## 
## Coefficients:
##              Estimate Std. Error z value Pr(>|z|)    
## (Intercept) -68.15417   11.71000  -5.820 5.88e-09 ***
## SEXM         -0.03079    0.62357  -0.049   0.9606    
## BPD           0.42833    0.16727   2.561   0.0104 *  
## OFD           0.02380    0.06676   0.356   0.7215    
## FL            0.36608    0.17408   2.103   0.0355 *  
## AC            0.08219    0.03847   2.136   0.0327 *  
## ---
## Signif. codes:  0 '***' 0.001 '**' 0.01 '*' 0.05 '.' 0.1 ' ' 1
## 
## (Dispersion parameter for binomial family taken to be 1)
## 
##     Null deviance: 785.087  on 630  degrees of freedom
## Residual deviance:  74.213  on 625  degrees of freedom
## AIC: 86.213
## 
## Number of Fisher Scoring iterations: 10
\end{verbatim}

\end{document}
